\chapter{总结与展望}
本文针对极化SAR图像目标检测算法中存在的计算量大,耗时长等问题提出了基于CPU+GPU异构架构的解决方案。
本文详细的分析了几种典型的基于分布和基于极化分解的SAR图像舰船检测方法,并分离其中可并行执行的部分,实现了基于
GPU的并行SAR图像舰船检测方法,本文完成的工作和创新点主要包括:

1)实现了基于GPU的极化白化滤波器极化SAR舰船目标检测算法,采用GPU多线程并行计算滑动窗内海杂波对应的
极化白化滤波器参数,并计算分割阈值得到检测结果。此外还合理的将数据与任务分配到了多块GPU上同时并行
计算,使得算法在时间效率进一步提升。2)实现了基于GPU的LMM舰船检测方法,该算法将估计海杂波分布参数的部分
附加到GPU上执行,并对参数迭代的过程使用GPU共享内存进行优化,实验结果表明基于GPU的LMM算法在时间性能上获得了数十倍的提升。
3)实现了基于GPU的PCDM舰船检测方法,该算法将PCDM矩阵与提取PCDM矩阵极化特征的计算部分放在GPU上并行计算。并行算法在
检测时间效率上获得了数十倍的提升,并且对复杂海况具有很好适应性。

本文在基于GPU的极化SAR图像检测上取得一定的成果,但是受限于时间、数据等原因,未来的工作还可以在以下几个方面展开:

1)在PCDM检测方法中,本文对于极化协方差差异矩阵的SPAN值与PSH特征采用经验阈值进行分割,可以尝试对这两个极化特征进行统计分析
以得到更加精准的自适应阈值,从而提升算法的检测精度。2)受限于CUDA提供的MATH API,较难实现Gamma分布,K分布等其它海杂波统计
模型,之后可以探究可在GPU上运算的更多统计分布模型,从而对海杂波分布拟合结果更加精准。3)算法基于linux系统下的CUDA运行环境进行设计,
并没有针对SAR系统实际的硬件环境来设计,未来可以结合含有嵌入式的GPU的SAR图像处理系统对算法在接口,时间,内存,功耗上做进行进一步的优化。

