\thusetup{
  %******************************
  % 注意:
  %   1. 配置里面不要出现空行
  %   2. 不需要的配置信息可以删除
  %******************************
  %
  %=====
  % 秘级
  %=====
  secretlevel={秘密},
  secretyear={10},
  %
  %=========
  % 中文信息
  %=========
  ctitle={基于CUDA的舰船目标检测算法设计},
  cdegree={工学硕士},
  cdepartment={电子工程系},
  cmajor={电子信息科学与技术},
  cauthor={孙天宇},
  csupervisor={杨健教授},
   % 日期自动使用当前时间,若需指定按如下方式修改:
  cdate={2020年5月15日},
  %
  %=========
  % 英文信息
  %=========
  etitle={An Introduction to \LaTeX{} Thesis Template of Tsinghua University v\version},
  % 这块比较复杂,需要分情况讨论:
  % 1. 学术型硕士
  %    edegree:必须为Master of Arts或Master of Science(注意大小写)
  %             “哲学、文学、历史学、法学、教育学、艺术学门类,公共管理学科
  %              填写Master of Arts,其它填写Master of Science”
  %    emajor:“获得一级学科授权的学科填写一级学科名称,其它填写二级学科名称”
  % 2. 专业型硕士
  %    edegree:“填写专业学位英文名称全称”
  %    emajor:“工程硕士填写工程领域,其它专业学位不填写此项”
  % 3. 学术型博士
  %    edegree:Doctor of Philosophy(注意大小写)
  %    emajor:“获得一级学科授权的学科填写一级学科名称,其它填写二级学科名称”
  % 4. 专业型博士
  %    edegree:“填写专业学位英文名称全称”
  %    emajor:不填写此项
  edegree={Doctor of Engineering},
  emajor={Computer Science and Technology},
  eauthor={Xue Ruini},
  esupervisor={Professor Zheng Weimin},
  eassosupervisor={Chen Wenguang},
  % 日期自动生成,若需指定按如下方式修改:
  % edate={December, 2005}
  %
  % 关键词用“英文逗号”分割
  ckeywords={舰船检测, 高性能计算, 极化白化滤波器, 极化协方差差异矩阵, 极化SAR},
  ekeywords={ship detection, CUDA, polarimetric whitening filter, polarimetric covariance difference matrix, polarimetric synthetic aperture radar}
}

% 定义中英文摘要和关键字
\begin{cabstract}
  合成孔径雷达(SAR)系统提供了一种全天候的遥感手段,可以在雷达波束照射下生成高分辨率的地物图像,
  目前已被广泛应用于海上目标检测与分类。为满足SAR图像舰船目标检测高时效性的要求,本文提出了基于
  CPU+GPU异构架构的SAR图像舰船目标检测方案。本文共实现了四种基于CPU+GPU异构架构的高效目标检测
  方法,总结如下:

  1)对于SAR图像,采用了对数混合高斯模型来描述幅值SAR图像中的非负海杂波分布,并采用最大期望(EM)方法
  来有效估计海杂波分布参数,根据事先给定的虚警率采用牛顿迭代法得到CAFR检测阈值。该方法在NVIDIA TITAN
  V GPU上的实验结果表明,相比于传统CPU平台,该方法在检测效率上获得了数十倍的提升。

  2)对于极化SAR图像,以极化白化滤波器(PWF)方法为例,实现了多GPU协同并行极化白化滤波器算法,该方法
  估计局部海杂波的协方差矩阵,并用其融合各个极化通道散射信息,得到一副相干斑抑制图像,对重构图像应用
  CFAR检测器得到最终检测结果。实验结果表明,该方法有效,并在检测时间上获得了显著的提升,明显优于传统的
  CPU平台。

  3)实现了基于GPU的反射对称性极化SAR图像舰船目标检测方法,该方法利用了海洋和人造金属物体反射对称性的差异,
  对散射矩阵中的HH通道与HV通道的相关性进行度量,构造了反射对称检测量。该检测量符合K分布,采用矩估计的方法
  对K分布参数进行估计,最后应用CFAR检测器得到检测结果。

  4)实现了基于GPU的极化协方差差异矩阵极化SAR图像舰船目标检测方法。该方法计算了每个像元与其周围3x3邻域的
  极化协方差矩阵的差值,由此得到极化协方差差异矩阵以改善局部区域的船-海对比度。为了充分利用极化协方差差异矩阵中
  的极化与强度特征,计算极化协方差差异矩阵的香浓熵,并应用阈值进行分割,得到最终检测结果。实验结果表明:相较于PWF方法,该方法
  检测准确率获得了有效的提升。

\end{cabstract}

% 如果习惯关键字跟在摘要文字后面,可以用直接命令来设置,如下:
% \ckeywords{\TeX, \LaTeX, CJK, 模板, 论文}

\begin{eabstract}
  Synthetic aperture radar (SAR) system provides an all-weather remote sensing method 
  that can generate high-resolution ground object images under radar beam irradiation. 
  Now, it has been widely used in Marine target detection and classification.In order 
  to meet the requirement of high timeliness of SAR ship target detection, this paper
  proposes a method based on CPU+GPU heterogeneous architecture for SAR image ship target 
  detection.In this paper, four kinds of efficient target detection based on CPU+GPU heterogeneous
  architecture are realized, these are summarized as follows:

  1)For SAR images, the log-mixture gaussian model is used to describe the non-negative 
  sea clutter distribution in amplitude SAR images, and the maximum expectation (EM) 
  method is used to effectively estimate the sea clutter distribution parameters. 
  According to the given false alarm rate in advance, the CAFR detection threshold is 
  obtained by Newton iteration method.The experimental results of this method on NVIDIA
  TITAN V GPU show that compared with the traditional CPU platform, the detection
  efficiency of this method is improved by dozens of times.

  2)For polarization SAR images, with polarization whitening filter (PWF) method as an example,
  this paper implements the GPU collaborative parallel polarization filter algorithm, the method 
  estimates the local sea clutter covariance matrix and combine the polarization scattering channel information to 
  get a pair of coherent spot suppression images. For the reconstruct image, CFAR detector was applied to get the 
  final detection results.The experimental results show that the method is effective and the detection 
  time is improved significantly, which is better than the traditional method
  CPU platform.

  3)A ship detection method using reflection-symmetric polarization features based on GPU 
  is implemented. This method uses the difference in reflection symmetry between ocean and 
  man-made metal objects and measure the correlation between the HH channel and HV channel in the scattering matrix 
  to construct the reflection symmetry quatity.  The detection quantity conforms to the K distribution. 
  The method of moment estimation is used to estimate the K distribution parameters.
   Finally, the CFAR detector is used to obtain the detection result.

  4)A ship detection method based on polarization covariance difference matrix is implemented.  This method calculates the polarization covariance matrix 
  difference between each pixel and its surrounding 3x3 neighborhood. The polarization covariance difference matrix improves
  the ship-sea contrast in the local area.  In order to make full use of the polarization and intensity features 
  in the polarization covariance difference matrix, the Shannon entropy of the polarization covariance
  difference matrix is ​​calculated, and the threshold is used for segmentation to obtain the final detection result.  
  Experimental results show that, compared with the PWF method, the detection accuracy of this method has been effectively improved.
\end{eabstract}

% \ekeywords{\TeX, \LaTeX, CJK, template, thesis}
